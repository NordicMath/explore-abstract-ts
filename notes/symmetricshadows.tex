
\section{Symmetric shadows}

\begin{definition}
  A \emph{symmetric shadow} is a triple $(R, S, \upsilon)$ where $R$ is a torsion-free $\lambda$-ring called the \emph{symmetry realm}, $S$ is a $\mathbb{Q}$-algebra called the \emph{shadow realm}, and $\upsilon: R \to S$ is a ring-homomorphism called the \emph{umbra}. 
  \begin{enumerate}
    \item An \emph{element} of the symmetric shadow refers to an element of the symmetry realm. 
    \item Finally, a symmetric shadow \emph{over} $S$ has $S$ as shadow realm, and one \emph{under} $R$ has $R$ as symmetry realm.
  \end{enumerate}
\end{definition}

\begin{definition}
  We define a morphism of symmetric shadows from $(R, S, \upsilon)$ to $(R', S', \upsilon')$ as a pair $(f, g)$ with $f : R \to R'$ a $\lambda$-ring homomorphism and $g : S \to S'$ a ring-homomorphism such that $\upsilon'(f(x)) = g(\upsilon(x))$. 
  \begin{enumerate}
    \item A morphism of symmetric shadows may be called an \emph{umbral morphism}. 
    \item Further, given a pair $(f, g)$, we call $f$ the \emph{symmetry part} and $g$ the \emph{shadow part}. 
    \item When the shadow part is the identity-morphism, we call the morphism \emph{anchored} (to $S$). 
    \item Composition of umbral morphisms is given by composition partwise. 
    \item A morphism is \emph{injective} when both parts are also injective. 
  \end{enumerate}
\end{definition}

\begin{remark}
  The category of symmetric shadows is clearly a comma-category
\end{remark}

\begin{definition}
  We define the univeral symmetric shadow $\Upsilon_R = (R^{\mathbb{N}}, R, f \mapsto f(1))$ where $R^{\mathbb{N}}$ is equipped with pointwise addition and multiplication and a adams operations corresponding to compression, meaning $\psi^k(f) : n \mapsto f(nk)$.
\end{definition}

\begin{propdef}
  For each symmetric shadow $\Gamma = (R, S, \upsilon)$ there is an anchored umbral morphism $\Psi_{\Gamma} : \Gamma \to \Upsilon_S$. The symmetry part is defined by $\Psi_{\Gamma}(r) = k \mapsto \upsilon(\psi^k(r))$. We call $\Psi_{\Gamma}(r)$ the $\psi$-points of $r$.
\end{propdef}


\begin{definition}
    We define a function $\Lambda^{tr} : R \to S^{\mathbb{N}}$, called the \emph{$\lambda$-points}, of an element in $R$, by the $k$-th index of $\Lambda^{tr}(r) = tr(\lambda^k(r))$. 
\end{definition}

\begin{proposition} 
  Establish arithmetic connection between the two
\end{proposition}

\begin{proposition}
  Given a morphism $(f, g) : (R, S, tr) \to (R', S', tr')$, we have $\Psi^{tr}(f(x)) = g^{\mathbb{N}}(\Psi^{tr}(x))$ and $\Lambda^{tr}(f(x)) = g^{\mathbb{N}}(\Lambda^{tr}(x))$.
\end{proposition}

\begin{definition}
  If $\Psi^{tr}$ is injective then we say $(R, S, tr)$ has faithful point counts. 
\end{definition}

\begin{definition}
  We define the \emph{Tannakian shadow} $TS(M \subset R)$ as the triple $(TS(M), R, tr)$, where $tr$ is the embedding into $TS(R)$ and then applying Tannakian symbol trace. 
\end{definition}

\begin{proposition}
  If $M$ is cancellable, then $TS(M \subset R)$ has faithful point counts. 
\end{proposition}
