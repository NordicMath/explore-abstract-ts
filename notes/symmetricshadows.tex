
\section{Symmetric shadows}

\begin{definition}
  A \emph{symmetric shadow} is a triple $(R, S, \upsilon)$ where $R$ is a torsion-free $\lambda$-ring called the \emph{symmetry realm}, $S$ is a commutative ring called the \emph{shadow realm}, and $\upsilon: R \to S$ is a ring-homomorphism called the \emph{umbra}. 
\end{definition}

\begin{definition}
  We define a morphism of symmetric shadows from $(R, S, \upsilon)$ to $(R', S', \upsilon')$ as a pair $(f, g)$ with $f : R \to R'$ a $\lambda$-ring homomorphism and $g : S \to S'$ a ring-homomorphism such that $\upsilon'(f(x)) = g(\upsilon(x))$. 
\end{definition}

\begin{remark}
  The category of symmetric shadows is clearly the comma-category $(F \downarrow Id_{CRing})$ where $F$ is the forgetful functor $Ring^{\lambda} \to CRing$
\end{remark}

\begin{definition}
  We define a function $\Psi^{tr} : R \to S^{\mathbb{N}}$, called the \emph{$\psi$-points}, of an element in $R$, by the $k$-th index of $\Psi^{tr}(r) = tr(\psi^k(r))$. If we equip $S^{\mathbb{N}}$ with a pointwise structure, it is clear that $\Psi^{tr}$ is a ring-homomorphism. 
\end{definition}

\begin{proposition}
  If we equip $S^{\mathbb{N}}$ with the compression-$\psi$ structure, where $\psi^k$ corresponds to $k$-compression, then $\Psi^{tr}$ is a $\lambda$-ring homomorphism.
\end{proposition}

\begin{definition}
    We define a function $\Lambda^{tr} : R \to S^{\mathbb{N}}$, called the \emph{$\lambda$-points}, of an element in $R$, by the $k$-th index of $\Lambda^{tr}(r) = tr(\lambda^k(r))$. 
\end{definition}

\begin{proposition} 
  Establish arithmetic connection between the two
\end{proposition}

\begin{proposition}
  Given a morphism $(f, g) : (R, S, tr) \to (R', S', tr')$, we have $\Psi^{tr}(f(x)) = g^{\mathbb{N}}(\Psi^{tr}(x))$ and $\Lambda^{tr}(f(x)) = g^{\mathbb{N}}(\Lambda^{tr}(x))$.
\end{proposition}

\begin{definition}
  If $\Psi^{tr}$ is injective then we say $(R, S, tr)$ has faithful point counts. 
\end{definition}

\begin{definition}
  We define the \emph{Tannakian shadow} $TS(M \subset R)$ as the triple $(TS(M), R, tr)$, where $tr$ is the embedding into $TS(R)$ and then applying Tannakian symbol trace. 
\end{definition}

\begin{proposition}
  If $M$ is cancellable, then $TS(M \subset R)$ has faithful point counts. 
\end{proposition}
